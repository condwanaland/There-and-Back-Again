\documentclass[]{book}
\usepackage{lmodern}
\usepackage{amssymb,amsmath}
\usepackage{ifxetex,ifluatex}
\usepackage{fixltx2e} % provides \textsubscript
\ifnum 0\ifxetex 1\fi\ifluatex 1\fi=0 % if pdftex
  \usepackage[T1]{fontenc}
  \usepackage[utf8]{inputenc}
\else % if luatex or xelatex
  \ifxetex
    \usepackage{mathspec}
  \else
    \usepackage{fontspec}
  \fi
  \defaultfontfeatures{Ligatures=TeX,Scale=MatchLowercase}
\fi
% use upquote if available, for straight quotes in verbatim environments
\IfFileExists{upquote.sty}{\usepackage{upquote}}{}
% use microtype if available
\IfFileExists{microtype.sty}{%
\usepackage[]{microtype}
\UseMicrotypeSet[protrusion]{basicmath} % disable protrusion for tt fonts
}{}
\PassOptionsToPackage{hyphens}{url} % url is loaded by hyperref
\usepackage[unicode=true]{hyperref}
\hypersetup{
            pdftitle={There and Back Again: Spatial and Temporal Variation in the Recruitment Dynamics of an Amphidromous Fish},
            pdfauthor={A thesis submitted to Victoria University of Wellington in partial fulfilment of the requirements for the degree of Master of Science in Ecology and Biodiversity; Victoria University of Wellington},
            pdfborder={0 0 0},
            breaklinks=true}
\urlstyle{same}  % don't use monospace font for urls
\usepackage{natbib}
\bibliographystyle{apalike}
\usepackage{longtable,booktabs}
% Fix footnotes in tables (requires footnote package)
\IfFileExists{footnote.sty}{\usepackage{footnote}\makesavenoteenv{long table}}{}
\usepackage{graphicx,grffile}
\makeatletter
\def\maxwidth{\ifdim\Gin@nat@width>\linewidth\linewidth\else\Gin@nat@width\fi}
\def\maxheight{\ifdim\Gin@nat@height>\textheight\textheight\else\Gin@nat@height\fi}
\makeatother
% Scale images if necessary, so that they will not overflow the page
% margins by default, and it is still possible to overwrite the defaults
% using explicit options in \includegraphics[width, height, ...]{}
\setkeys{Gin}{width=\maxwidth,height=\maxheight,keepaspectratio}
\IfFileExists{parskip.sty}{%
\usepackage{parskip}
}{% else
\setlength{\parindent}{0pt}
\setlength{\parskip}{6pt plus 2pt minus 1pt}
}
\setlength{\emergencystretch}{3em}  % prevent overfull lines
\providecommand{\tightlist}{%
  \setlength{\itemsep}{0pt}\setlength{\parskip}{0pt}}
\setcounter{secnumdepth}{5}
% Redefines (sub)paragraphs to behave more like sections
\ifx\paragraph\undefined\else
\let\oldparagraph\paragraph
\renewcommand{\paragraph}[1]{\oldparagraph{#1}\mbox{}}
\fi
\ifx\subparagraph\undefined\else
\let\oldsubparagraph\subparagraph
\renewcommand{\subparagraph}[1]{\oldsubparagraph{#1}\mbox{}}
\fi

% set default figure placement to htbp
\makeatletter
\def\fps@figure{htbp}
\makeatother

\usepackage{booktabs}
\usepackage{etoolbox}
\makeatletter
\providecommand{\subtitle}[1]{% add subtitle to \maketitle
  \apptocmd{\@title}{\par {\large #1 \par}}{}{}
}
\makeatother

\title{There and Back Again: Spatial and Temporal Variation in the Recruitment
Dynamics of an Amphidromous Fish}
\providecommand{\subtitle}[1]{}
\subtitle{Conor Neilson}
\author{A thesis submitted to Victoria University of Wellington in partial
fulfilment of the requirements for the degree of Master of Science in
Ecology and Biodiversity \and Victoria University of Wellington}
\date{2016}

\begin{document}
\maketitle

{
\setcounter{tocdepth}{1}
\tableofcontents
}
\chapter{Preface}\label{intro}

\begin{quote}
\emph{One planet, one experiment}

\emph{- E.O. Wilson}
\end{quote}

\section{Abstract}\label{abstract}

A primary goal of ecology is to identify the factors underlying
recruitment variability, and how they may shape population dynamics.
Recruitment is driven by the input of new individuals into a population.
However, these individuals often show high diversity in phenotypic
traits and life histories, and the consequences of this variation are
poorly understood. Phenotypic variation is widespread among the early
life stages of fish, and this variation may be influenced by events
occurring across multiple life stages. While many studies have
investigated phenotypic variation and its effect on population dynamics,
comparatively few studies use an integrated approach that evaluates
patterns and processes across multiple life history stages. Here I focus
on a native amphidromous fish, \emph{Galaxias maculatus}, and I explore
patterns and consequences of phenotypic variation during larval stages,
migratory stages, and post-settlement stages of this fish.

I explore variability in phenotypes and early life history traits of G.
maculatus through both space and time. I use metrics derived from body
size and otolith-based demographic reconstructions to quantify
potentially important early life history traits. I found that cohorts of
juvenile fish sampled later in the year were comprised of individuals
that were older, smaller, and grew more slowly relative to fish sampled
earlier in the year. I also found that two sampled sites (the Hutt River
and the Wainuiomata River) showed different temporal trends, despite
their close geographical proximity.

I then investigated whether phenotype was related to mortality. I used
otolith-based traits to characterise larval `quality' for individual
fish. I then calculated the average larval quality for discrete cohorts
of fish, and used catch-curve analysis to estimate mortality rates for
these cohorts. I investigated the overall relationship between quality
and mortality, and compared the trend between two sites. My results
indicate that phenotype and mortality were not significantly correlated.
However, this inference may be limited by low statistical power; the
non-significant trends suggest that the relationship might be negative
(i.e., larvae of higher quality tend to have lower rates of mortality).
This trend is typical of systems where population expansion is limited
by food rather than predators.

I then investigated whether phenotypic traits in the juvenile cohorts
were correlated with traits in adult cohorts. I resampled the focal
populations \textasciitilde{}6 months after sampling the juvenile stages
(i.e., targeting fish from sampled cohorts that had survived to
adulthood), and I used data from otoliths to reconstruct life history
traits (hatch dates and growth histories). I compared adult life history
traits to the traits of discrete juvenile cohorts.

My results suggest that fish that survived to adulthood had
comparatively slower growth rates (reconstructed for a period of
larval/juvenile growth) relative to the sampled juvenile cohorts (where
growth rate was estimated for the same period in their life history). I
also found that the distributions of hatch dates varied between sites.
Fish that survived to adulthood at one site hatched later in the
breeding season, while adult stages from the other site had hatch dates
that were distributed across the entire breeding season. Both hatch date
and growth rate are likely linked to fitness, and their interaction may
have influenced patterns of survival to adulthood. These results provide
evidence for carry-over effects of larval phenotype on juvenile success.

Collectively my thesis emphasises the importance of phenotype and life
history variability in studies of recruitment. It also highlights the
importance of spatial scale, and how biological patterns may differ
between geographically close systems. Some of the general inferences
from my study may extend to other migratory Galaxiid species, and
perhaps more generally, to many species with extensive larval dispersal.
Finally, my work highlights potentially important interactions between
phenotypes, life histories, and mortality, which can ultimately shape
recruitment, and the dynamics of populations.

\section{Acknowledgements}\label{acknowledgements}

I always suspected that the acknowledgements section would end up being
the longest section of this thesis. In truth, there has been a
phenomenal amount of people who have contributed to this in some way,
and it wouldn't feel right if I didn't thank you all.

First and foremost, I want to thank my supervisor, Jeff Shima. Jeff,
thank you for everything you've done for me over the past two years. You
have helped me to grow and develop as a scientist, and your input has
always been appreciated. Thank you especially for reigning in the first
thesis plan I submitted to you. That would have kept me working until
2020! My gratitude also goes out to the members of the Shima Lab. Thanks
for listening to me rabbit on about whitebait, and for providing support
and advice.

To the VUCEL community, I've really enjoyed being a part of this group
of people. Cheers for the BBQs, the morning teas, and the general
get-togethers. You've all made my Master's a fantastic experience. John,
Dan, and Snout, thank you for all the technical assistance. Everyone
would be lost without you three!

This thesis wouldn't have been possible without the small army of
volunteers I had come and assist with whitebaiting. In no particular
order, thank you to Kayla, Tory, Savita, Heyes, Andrew, Chris, Vinnie,
Jessie, Ali, Mel, Eden, Emily, Anna, Jordan, James, and Max. I also want
to thank John, Danny, Tom, Kelly, and Jim for donating samples and
general advice on whitebaiting.

Chris, Jess, and Vinnie, thanks for being my partners in crime during
this journey. It's been great to collaborate, share data, and tackle
Galaxiid ecology as a team. Cheers for listening to my ridiculous
experimental ideas, and stopping me using models that no normal human
would run. Vinnie, thank you in particular for your incredible amount of
help in the field. You made me keep going when I was ready to give up,
and kept on pushing when everything kept going wrong.

To all of my friends, and particularly my flatmates, thank you for
understanding why I neglected you. Your support has meant the world to
me. Thanks also needs to be said to Alex, for getting me out of a tight
spot, Ben, for some much needed advice, Lisa Woods, who knows more about
statistics than anyone I've met, and Phoebe, for answering seemingly
endless questions about everything.

Chris, this concludes five years of us studying together. Thank you for
always being there as a source of advice, ideas, and generally helping
me to feel better when everything goes wrong. I'm going to miss working
alongside you.

There are three people in particular I need to mention. Snout, thank you
so much for your guidance. This thesis never would have got here without
you. Your knowledge of logistics, fieldwork, otoliths, and everything in
between has been invaluable to me, and I cannot thank you enough for all
your patience. Also, your cooking skills are second to none! Secondly, I
owe a huge debt of gratitude to Mark Kaemingk. Mark, you have been like
a second supervisor to me. You introduced me to whitebait, and you have
totally changed the way I think about science. This thesis has been
shaped by you in so many ways, and it has been a true pleasure having
you as a mentor and friend.

And to my partner Elyse. Thank you for all your love and support. You
may have no interest in fish population ecology, but you understood my
passion, and always encouraged and supported me.

Lastly, I want to say a massive thank you to my parents, Ian and Vicki.
You have always supported me in whatever path I chose to pursue, and for
that I am thankful.

\chapter{Literature}\label{literature}

Here is a review of existing methods.

\chapter{Methods}\label{methods}

We describe our methods in this chapter.

\chapter{Applications}\label{applications}

Some \emph{significant} applications are demonstrated in this chapter.

\section{Example one}\label{example-one}

\section{Example two}\label{example-two}

\chapter{Final Words}\label{final-words}

We have finished a nice book.

\bibliography{book.bib,packages.bib}

\end{document}
